\documentclass[letterpaper, 12pt]{article}
\usepackage{amsmath, amssymb}
\usepackage[top=2in, bottom=1.5in, left=1in, right=1in]{geometry}
\begin{document}
\title{Forces}
\author{Mr. H. van Bemmel}
\date{March 20, 2013}
\maketitle


\section{Sir Isaac Newton}
Newton was born on December 25, 1642 in Woolsthorpe, England. He went to King's School
in Grantham, and went to Cambridge at Trinity College.

In the mid-1660s, a great blague occured. The rich left the town leaving the
poor and the gates of the city were locked to the sick. Cats and dogs were destroyed
thinking they were spreading the disease, which reduced predation on the rats and mice.
Because of this, Newton was sent to his mother's farm.
\paragraph*{Newton's Developments}
\begin{itemize}
    \item Light diffraction
    \item Prisms
    \item Laws of gravitation and motion
    \item Reflecting telescopes
    \item Calculus
\end{itemize}
\subsection{After The Principia}
Halle realized the comet named after him recurred every ~70 years, and approached
Newton asking for an explanation for it. Halle realized that Newton's physics
was completely new and realized the need to publish it. It published as
the Principia, and all of the proofs held within were done geometrically
in the style of Euclid.

In 1963, he joined the Royal Society and became its president in 1703.
He published his Optiks in 1703, right before he had a breakdown which could
have been a stroke or menigitis.

In 1696 he became a sinecure for the Tower and a master in 1699, worked on the
metallurgical aspects of coinage. He aggressively pursued counterfeiters
and clippers, and invented milling on the edge of coins.

He spent much time in his later years on alchemy, and he may have uptook a lot
of mercury or lead. He died in 1727 due to a kidney stone infection.

\section{Forces ($\vec{F}$)}
In common understanding, a force is a push or pull. The effect of forces was
first described by Sir Isaac Newton. He postulated 3 (4) laws of forces which
remain relevant today.
\begin{enumerate}
    \item \textbf{Law of Intertia}:\\
    ``An object moving at a constant velocity shall remain so until acted
    upon by an external force.''
    \item The vector sum of all forces acting on an object shall
    effect an acceleration $\vec{a}$ of that object which is inversely
    proportional to the object's mass.
    \[
        \begin{array}{lrl}
        \sum^n{F_k} & = & m\vec{a}_\textrm{observed}\\    
        & = & \vec{F}_\textrm{net}
        \end{array}
    \]
    \item \textbf{Actions / Reaction Law}:\\
    An action is a force in archaeic English. Where there exists an $\vec{F}$,
    there shall also exist $-\vec{F}$ an equal an opposite force.\\
    Huh? Why then does anything move?
\end{enumerate}

\end{document}
