\documentclass[letterpaper]{article}
\usepackage[margin=1in]{geometry}
\usepackage{graphicx}
\usepackage{amsmath, amssymb}
\begin{document}
\title{1D Kinematics}
\author{Mr. H. van Bemmel}
\date{February 19, 2013}
\maketitle

The study of motion without consideration for the study of this motion.

Motion is: the change of an object's position over an arbitrary time period.

To study motion, we require a dependable description of position.
Such a description must have:
\begin{itemize}
    \item a constant and mutually-agreed fiducial point (origin)
    \item a constant and mutually-agreed set of axes
\end{itemize}
The assembly of the fiducial point and these axes is called a {\bf Frame of Reference} (FoR or simply ``frame").

To locate an object in a frame requires a distance and a direction from the fiducial.

\section{Vectors}
Any variable which requires a magnitude and a direction to make sense is called a vector.
Example:
\begin {itemize}
    \item Force
\end{itemize}

A quantity that has only magnitude is called a scalar. For example: 
\begin{itemize}
    \item mass
    \item temperature
    \item money
\end{itemize}

We will denote a vector $x$ with an arrow above the letter: $\vec{x}$.

\subsection{Stating Vectors}
An example:
    Let us say we have a vector going on a 2D cartesian plane going from the origin
    to (3, 4). We have two basic coordinate systems to choose from:
    \begin{enumerate}
        \item Polar, in which we have an angle $\theta$ and a radius (magnitude) $r$.
        \item Cartesian or rectangular, in which we have an ordered pair $(x, y)$.
    \end{enumerate}

For polar form, {\bf all} $\theta's$ are referenced from the $+x$ axis.
    (Counterclockwise).
As long as $\theta$ is referenced from the $+x$ axis, your $x, y$ components
will always be correct. In polar form, the magnitude is {\bf always} positive.

\subsubsection{Other angle issues}
\paragraph{Bearings}
Some of our kinematics work will involve directions. A compass direction is
called a bearing ($\gamma$) and is measured clockwise from $+y$.
However, components of such vectors require a $\theta$ reference counterclockwise
from $+x$.
Conversion:
\[
    (\theta = 450 - \gamma) \in [0, 360)
\]
\paragraph{Types}
\begin{itemize}
    \item Absolute\hfill\\Bearing is an angle clockwise from $+y \in [0, 360)$
    \\ Very typical for professional navigation.
    \item Relative\hfill\\Suppose $\gamma = 128^{\circ}$
    
    We can also produce a bearing like this:
    \begin{enumerate}
        \item Choose the cardinal direction (N, E, S, W) to which this $\gamma$
        most closest points.
        \[
            S\hfill 42^{\circ} E
        \]
        \item Correct is so many degrees on either side of the chosen cardinal.
        For N or S, it is E or W, and vice-versa.
    \end{enumerate}
\paragraph{Two basic rules}
\begin{enumerate}
    \item If correction is $45^\circ$ use NE, NW, etc
    \item If correction is $<45^\circ$, use trig and common sense.
\end{enumerate}
\end{itemize}
\subsection{$\mathbb{R}^3$ - Spatial Motion}
Can use $(x, y, z)$, or $(r, \theta, \phi)$, spherical, or $(r, \theta, h)$, cylindrical.\\
$\theta \in [0, 360) \textrm{ or } [-180, 180)$\\
$\phi \in [-90, 90]$\\

\subsection{$\mathbb{R}^3$}
Above $\mathbb{R}^3$ only n-tuples are defined. We cannot visualize them and
must trust the theory. 

\subsection{Lines}

For $y = mx$, which passes through the origin, $\theta =\textrm{ constant}$.

For $y = mx + b$, $r = \frac{b}{\sin{\theta} - m\cos{\theta}}$ (derivation will
be left as an exercise to the reader.)

\section{Position Vectors($\vec{r}$)}
To analyze motion, it is sessential to locate objects in a frame. A position
vector begins at the origin and ends at the object. If the object moves
 the $\vec{r}$ instantaneously moves to follow it.

\section{Displacement($\vec{s}$)}
If an object moves, there is a straight vector that points from the start to the
finish. This is called {\bf displacement}. It is the net change of position.

In addition or subtraction, vector variable algebra is identical to $\mathbb{R}$.

\begin{itemize}
    \item Commutivity. $a + b = b + a$
    \item Associativity. $(a + b) + c = a + (b + c)$
\end{itemize}

If we move to point A, then to point B, we can simply move to point B.

Unit vectors are vectors with magnitudes of 1.
We can represent vectors as a scalar multiplied by a unit vector:
$\vec{a} = a \hat{a}$.

There are three special unit vectors: $\hat{\imath}, \hat{\jmath}, \hat{k}$, which
represent vectors in $x, y, z$ respectively. Therefore, a rectangular
coordinate in three dimensions $(x, y, z)$ can be represented as
$x\hat{\imath} + y\hat{\jmath} + z\hat{k}$.

\subsection{Distance ($d$)}
The length of the actual path taken is called distance.
Since distance is a scalar and is accumulative, the distance is always
greater than or equal to zero.

\subsection{Vector ``Locations"}

A vector is defined only by its magnitude \& direction, never its position.
As such, parallel vectors are not only identical and indistinguishable. 
The direction of a vector can be mathematically described as the slope.

\[
    y = mx + b
\]

So $y$ is a family of all lines of slope $m$.

\section{Rates}

A rate is a comparison of a variable and time, typically standardized as
{\bf unit time}.

\begin{table}[ht]\label{SiUnitsTable}
\centering
\caption{SI Base Units}



\begin{tabular}{l| c r} 
    & mks & cgs\\ \hline
    Length & m & cm\\
    Time & s & s\\
    Mass & kg & g\\
    Temperature & K & K\\
    Force & N & dynes\\
    Energy & J & ergs\\
\end{tabular}
\end{table}

\subsection{Velocity ($\vec{v}$)}

The rate of change of $\vec{s}$ is defined as the velocity:
\[
    \vec{v}_{ANG} = \frac{\Delta{\vec{s}}}{\Delta{t}}
\]

If $\Delta{t} \rightarrow 0$ we obtain a limit
\[
    \vec{v}_{INST} = \lim_{\Delta{t} \rightarrow 0} {\frac{\Delta{s}}{\Delta{t}}}
\]
\[
    \vec{v} = \frac{d\vec{s}}{dt}
\]

\subsection{Acceleration($\vec{a}$)}

The rate of change of $\vec{v}$ is called {\em acceleration}.
\[
    \vec{a}_{AVG} = \frac{\Delta{\vec{v}}}{\Delta{t}}
\]

If $\Delta{t} \rightarrow 0$ then another limit,
\[
    \vec{a} = \lim_{\Delta{t} \rightarrow 0} {\frac{\Delta{\vec{v}}}{\Delta{t}}}
\]
\[
    \vec{a} = \frac{d\vec{v}}{dt}
\]

\subsection{Higher rates}
\begin{enumerate}
    \item Surge or jerk, in the UK and US respectively 
    \item Snap
    \item Crackle
    \item Pop
\end{enumerate}

\section{Mathematical Relationships in Kinematics}

The interrelations between kinematics variables are firstly derived
from definitions and then subject to mathematical rigour. Recall:
\[
    a = \frac{\text{d}v}{\text{d}t}
\]
Hence,
\[
    \text{d}v = a\text{d}t
\]
\[
    \int{\text{d}v} = \int{a\text{d}t}
\]
\[
    v = at + c
\]

The $c$ represents $v(0)$. It is practise in physics to write $v(0) = v_0$
seen as an intitial condition. We tend to write polynomials with their terms
oriented in order of increasing exponents. 
This is because it is typical of
a Taylor series and they are used extensively to approximate more involved
functions. Therefore, our equation becomes
\begin{equation} \label{v}
    v = v_0 + at
\end{equation}

Our second equation comes from another definition.
\[
    v = \frac{\text{d}s}{\text{d}t}
\]
\[
    \text{d}s = v\text{d}t
\]
\[
    \text{d}s = (v_0 + at)\text{d}t
\]
\[
    = v_0\text{d}t + at\text{d}t
\]

We arrive at:
\begin{equation} \label{s}
    s = s_0 + v_0 t + \frac{1}{2}at^2
\end{equation}
\paragraph*{Some constraints:}
\begin{enumerate}
    \item $a$ is assumed constant over an interval.
    \item Time intervals must be changed if:
    \begin{itemize}
        \item $\vec{a}$ changes
        \item $\vec{v}$ changes sign
        \item $\vec{v}$ changes arbitrarily (magic)
    \end{itemize}
    \item Question dictates it.
\end{enumerate}
\newpage
In (\ref{v}), $t = \frac{v - v_0}{a}$ and substitute this into (\ref{s}).
\[
    s = s_0 + v_0 \left(\frac{v - v_0}{a}\right) + 
    \frac{1}{2} a \left( \frac{v - v_0}{a}\right) ^ 2
\]
\[
    \vdots
\]
\begin{equation} \label{eq3}
    v^2 = v_0^2 + 2as
\end{equation}
\subsection{Reminder}
\paragraph*{(\ref{eq3}) is only valid iff:}
    \begin{itemize}
        \item $v$ has a constant sign.
        \item Single time interval.
    \end{itemize}
\subsection{Caution}
$\vec{v}_{AVG}$ is defined as 
\[
\vec{v}_{AVG} = \frac{\sum{\vec{s}}}{\sum{t}}
\]
and
\[
    v_{AVG} = \frac{\sum{d}}{\sum{t}}
\]
But:
\[
    v_{AVG} \neq \frac{\sum{v_k}}{\sum{t}}
\]
\section{Graphing kinematics equations}
An object starting from rest accelerates at a constant rate until
the object has moved 4m. It then stops for 1s.
It then goes at $-2\text{ms}^{-1}$ for 2s and finally $\vec{a}$ at 1ms$^{-2}$
for 2.
An exercise to the reader:
Cretae a table.

\end{document}
