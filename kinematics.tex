\documentclass{article}
\usepackage{graphicx}
\begin{document}
\title{Kinematics}
\author{Mr. H. van Bemmel}
\date{February 19, 2013}
\maketitle

The study of motion without consideration for the study of this motion.

Motion is: the change of an object's position over an arbitrary time period.

To study motion, we require a dependable description of position.
Such a description must have:
\begin{itemize}
    \item a constant and mutually-agreed fiducial point (origin)
    \item a constant and mutually-agreed set of axes
\end{itemize}
The assembly of the fiducial point and these axes is called a {\bf Frame of Reference} (FoR or simply ``frame").

To locate an object in a frame requires a distance and a direction from the fiducial.

\subsection{Vectors}
Any variable which requires a magnitude and a direction to make sense is called a vector.
Example:
\begin {itemize}
    \item Force
\end{itemize}

A quantity that has only magnitude is called a scalar. For example: 
\begin{itemize}
    \item mass
    \item temperature
    \item money
\end{itemize}

We will denote a vector $x$ with an arrow above the letter: $\vec{x}$.

\subsubsection{Stating Vectors}
An example:
    Let us say we have a vector going on a 2D cartesian plane going from the origin
    to (3, 4). We have two basic coordinate systems to choose from:
    \begin{enumerate}
        \item Polar, in which we have an angle $\theta$ and a radius (magnitude) $r$.
        \item Cartesian or rectangular, in which we have an ordered pair $(x, y)$.
    \end{enumerate}
    
\end{document}
